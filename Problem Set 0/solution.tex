\documentclass{article}
\usepackage[margin=1in]{geometry}
\usepackage{amsmath, amssymb, amsfonts, mathtools, fancyhdr, tikz, cancel, enumitem, lastpage}
\usepackage{libertinus}

\definecolor{pen}  {RGB}{36,  41, 46}
\definecolor{paper}{RGB}{197, 186, 175}
        
\definecolor{one}{RGB}{124, 87 , 53 }
\definecolor{two}{RGB}{153, 94 , 65 }
\definecolor{thr}{RGB}{174, 130, 87 }
\definecolor{for}{RGB}{124, 53 , 54 }
\definecolor{fiv}{RGB}{124, 123, 53 }
\definecolor{six}{RGB}{87 , 53 , 124} 
       

\pagecolor{paper}
\color{pen}

\pagestyle{fancy}
\renewcommand{\footrulewidth}{0.4pt}
\fancyhead[C]{\textbf{Matematical Cryptography}}
\fancyhead[L]{  St.Name:Ali Zare}
\fancyhead[R]{  Problem Set: 0 }
\fancyfoot[C]{\thepage/\pageref{LastPage}}
\fancyfoot[L]{Al}
\fancyfoot[R]{IZ}
\linespread{1.2}
\begin{document}
\begin{itemize}
	\item{\textcolor{for}{\textbf{Problem 1}}}
		\begin{enumerate}[label=(\alph*)]
		\item 
			\begin{alignat*}{5}
				30030 &= 257\ & ( & 116 ) &+&\ 218 \\
				257   &= 218\ & ( & 1 )   &+&\ 39  \\
				218   &= 39\  & ( & 5 )   &+&\ 23  \\
				39    &= 23\  & ( & 1 )   &+&\ 16  \\
				23    &= 16\  & ( & 1 )   &+&\ 7   \\
				16    &= 7\   & ( & 2 )   &+&\ 2   \\
				7     &= 2\   & ( & 3 )   &+&\ 1
			\end{alignat*}
			Which leads us to the fact that $gcd(30030, 257)=1$. by assumption we have the
			unique prime factorization of $30030=2 \times 3 \times 5 \times 7 \times 11 \times 13$.
			By definition of $gcd$ we know that any of $A=\{2, 3, 5, 7, 11, 13\}$ is not a prime factor for $257$.
			and on ther other hand we have the fact that $\sqrt{257}\approx 16$ and non of primes up to $16$ 
			that are elements of set $A$ divide
			$257$ so there could not be any prime factor for $257$ which yields that $257$ is prime $\blacksquare$
		\item 
		\begin{alignat*}{5}
				4883 &= 4369\ & ( & 1 )  &+&\ 514 \\
				4369 &= 514\  & ( & 8 )  &+&\ 257  \\
				514  &= 257\  & ( & 2 )  &+&\ 0  \\
		\end{alignat*}
		By part a we know that $257$ is prime and thus we factor numbers as follows.
		$\tfrac{4883}{257}=19$, and because $19$ is prime, for the factorization we have: $4883 = 257 \times 19$.
		With the same argue for $4369$ we have the factorization, $4369 = 257 \times 17$.
		\end{enumerate}
	\item{\textcolor{for}{\textbf{Problem 2}}}
	\begin{enumerate}[label=(\alph*)]
		\item
			We know that every linear combination of $a$ and $b$ in the form $ax+by$ for $x,y\in \mathbb{Z}$ is a multiple of
			their greatest common divisor $gcd(a,b)$, by knowing this fact and that there is a linear combination of $a,b$ 
			such that $ax+by=1$ it is trivial that $gcd(a,b)=1$ $\blacksquare$
		\item 
			Assume that $a$ is invertible mod $b$ then the equation $ax\stackrel{b}{\equiv}1$ has a solution for $x$, which by definition of 
			modular equation leads us to the equation $ax=by+1$ for some value of $b$, which means that $ax+by=1$ for some values of
			$x$ and $y$ which means that $gcd(a,b)=1$.\\
			Conversely assume that $gcd(a,b)=1$ which means that $ax+by=1$ for some value of $x$ and $y$ now consider this equation modular
			$b$ which gives us $\overline{ax}+\cancel{by} \stackrel{b}{\equiv} \overline{1}$ which means $a$ is invertible modular $b$ 
			$\blacksquare$
		\item
			\begin{align*}
				101 &= 17\ (5) + 16  &&(\text{I}) \\
				17  &= 16\ (1) + 1 && (\text{II})
			\end{align*}
			By the equation (II) we can write $16$ as follows $16 = 17 - 1$ by putting this into the equation (I) we have the equation
			$101 = 17(5) + \left(17 - 1 \right)$ which leads us to $101 = 17(6) - 1$, Thus $ 1 = 17(6) - 101(1) $. Hence $17^{-1} mod(101) = 6$ 
			$\blacksquare$
	\end{enumerate}
	\item{\textcolor{for}{\textbf{Problem 3}}}
		\begin{enumerate}[label=(\alph*)]
		\item
		\begin{align*}
			\left\{ \begin{array}{l}
				x \stackrel{4}{\equiv} 1 \\
				x \stackrel{6}{\equiv} 2
			\end{array}\right. \longrightarrow gcd(4,6)=2
		\end{align*}
		Assume for the sake of contradiction that there is a solution for this system. Means that there is a $x$ such that.
		\begin{align*}
			\left\{\begin{array}{l}
					x\stackrel{4}{\equiv}1\Rightarrow x=4k+1\ (k\in\mathbb{Z}) \\ 
					x\stackrel{6}{\equiv}2\Rightarrow x=6l+2\ (l\in\mathbb{Z})
			\end{array}\right.
			\Rightarrow 4k+1=6l+2\Rightarrow 4k=6l+1 
		\end{align*}
		But in the last equation the Left-Hand-Side is always \underline{even} and the Right-Hand-Side is always \underline{odd}
		which contradicts. Thus there is no such a solution for the system this completes the proof of our counterexampler~ $\blacksquare$
		\item
			\begin{align*}
				\left\{\begin{array}{l}
						x\stackrel{17}{\equiv}2 \\
						x\stackrel{101}{\equiv}9
					\end{array}\right.
			\end{align*}
			By the first equation we have $x=17k+2$, then by putting this together with the second equation 
			we have that $17k+2\stackrel{101}{\equiv}9$ which leads us to $17k\stackrel{101}{\equiv}7$. Since we have already 
			calculated the inverse of $17$ modulo $101$ we can now evaluate the value of $k$ modular $101$.
			\begin{align*}
				6 \times 17k \stackrel{101}{\equiv} 6 \times 7 \Rightarrow k \stackrel{101}{\equiv} 42 \Rightarrow
				k = 101u+42
			\end{align*}
			Now by putting this together with the last equation we hae $x=17(101u+42)+2$. Thus $x=1717u+716$~$\blacksquare$
		\end{enumerate}
	\item{\textcolor{for}{\textbf{Problem 4}}}
		\begin{enumerate}[label=(\alph*)]

			\item Assume for the sake of contradiction that there are two identity elements $e$ and $\acute{e}$, we get the contradiction by just
			  applying the definition of identity element on each:
			  \begin{align*}
				m\left(e,\acute{e}\right) =
				\left\{\begin{array}{ll}
						e & \acute{e}\text{ is identity} \\
						\acute{e} & e \text{ is identity}
					\end{array}\right.
				\Longrightarrow e=\acute{e}
			  \end{align*}
			  Which contradict so the identity element must be unique $\blacksquare$\\[5pt]
			  Assume for the sake of contradiction 
			  that arbitrary element $x$ of our group has two inverses $x_{1}^{-1}$ and $x_{2}^{-1}$.
			  \begin{align*}
				\left\{\begin{array}{ll}
						m(x, x_{1}^{-1}) = e \\
						m(x, x_{2}^{-1}) = e
				\end{array}\right. \xrightarrow[\text{identity element}]{\text{uniqueness of}}
				m(x,x_{1}^{-1}) = m(x,x_{2}^{-1}) \Rightarrow x_{1}^{-1} = x_{2}^{-1} = x^{-1}
			  \end{align*}

		        \item We need to show (I)-Injectivity and (II)-Surjectivity
			  \begin{enumerate}[label=(\Roman*)]
				  \item  Assume that $hg = \acute{h}g$ by multiplying $g^{-1}$ to the both sides of equation we have $h=\acute{h}$.
				  \item Take arbitrary element $h\in G$ we claim that $g^{-1}h$ is the element that maps under the function $m_g(h)$ to
					  the element $h$. To prove the claim we have $m_g(g^{-1}h) = gg^{-1}h = h$ as desired.
			  \end{enumerate}
			  So function is both injective and surjective together means that the function is Bijection $\blacksquare$
			  \newpage

		        \item
			  \begin{enumerate}[label=(\Roman*)]
				  \item Closed under operator: This is easy to see that $ax+bz, \cdots , cy+dt$ are real numbers\\
					  since $a,b,c,d,x,y,z,t\in\mathbb{R}$
					\begin{align*}
						A \times  B = 
						\left(\begin{matrix} a & b \\ c & d \end{matrix}\right) \times 
						\left(\begin{matrix} x & y \\ z & t \end{matrix}\right) \Rightarrow
						\left\{\begin{array}{l}
						A\in GL_{2}(\mathbb{R})\\
						B\in GL_2(\mathbb{R})
					\end{array}\right.
					\Rightarrow A \times B \in GL_2(\mathbb{R})
					\end{align*}
				  \item 
				  	Existence of Inverse: This follows immediately from the fact the determinant is non-zero and basic Linear-Algebra.
					Identity element is just identity matrix $I_2$ 
					\begin{align*}
						&A = \left(\begin{matrix} a & b \\ c & d \end{matrix}\right)
						&&A^{-1} = \frac{1}{ad-bc}\left(\begin{matrix} d & -b \\ -c & a \end{matrix}\right)=
						\frac{1}{1} \left(\begin{matrix} d & -b \\ -c & a \end{matrix}\right)=
						\left(\begin{matrix} d & -b \\ -c & a \end{matrix}\right)
					\end{align*}

				  \item 
					$(A \times B) \times C = A \times (B \times C)$: This is easy to verify by Linear-Algebra. Just write it down 
					it is straightforward.
			  \end{enumerate}
			  This is not a cummutative group. Counterexample is as follows:
			  \begin{align*}
				&\left(\begin{matrix} 1 & -1 \\ 1 & 1 \end{matrix}\right) \left(\begin{matrix} 2 & 1 \\ 1 & 1 \end{matrix}\right)
				= \left(\begin{matrix} 1 & 0 \\ 3 & 2 \end{matrix}\right)
				&&
				\left(\begin{matrix} 2 & 1 \\ 1 & 1 \end{matrix}\right) \left(\begin{matrix} 1 & -1 \\ 1 & 1 \end{matrix}\right)=
				\left(\begin{matrix} 3 & -1 \\ 2 & 0 \end{matrix}\right)	
			  \end{align*}

			\item
				  \begin{enumerate}[label=(\Roman*)]
					  \item Closed under operator: $\overline{a} \times  \overline{b} = \overline{a \times b}$, Thus it is closed.
					  \item Identity element is : $\overline{1}$ and inverse for every element exists since $p$ is considered to be prime:
						$gcd(p, a)$ is either $p$ which means $p$ divide $a$, This is not the case because every element of our 
						group is to be considered lower than $p$, or $gcd(p,a)=1$ which by bezout's coefficients we know that 
						there is an inverse for $a$ modulo $p$.
					\item $\overline{a} \times \left(\overline{b} \times \overline{c}\right) = \left(\overline{a} \times \overline{b}\right) \times \overline{c}$: which is trivially true by the definition.\\
				  \end{enumerate}
				  Under addition is similar.

			\item No it is not closed under operator: $\overline{3} \times \overline{5} = \overline{0}$\\
				  The largest prime number before $15$ is $13$ so $\frac{\mathbb{Z}}{13\mathbb{Z}}\equiv \mathbb{F}_{13}$ 
				  is the maximal subset of 
				$\frac{\mathbb{Z}}{15\mathbb{Z}}$ which is a group under multiplication.

			\item We know that $\varphi$ is a multiplicative function. Let's evaluate the $\varphi$ for prime values:
				\begin{align*}
						\varphi(p) &= \#\{1,\cdots,p-1\} = p-1\\
						\varphi(p^{\alpha}) &= \# \{1,\cdots,p-1,p+1,\cdots,p^2-1,p^2+1,\cdots,p^{\alpha}-1\}=
						p^{\alpha-1}(p-1)
				\end{align*}
				Now let's evaluate the $\varphi$ for arbitrary $N$ with the factorization $N=p_1^{\alpha_{1}}\cdots p_s^{\alpha_{s}}$.
				\begin{align*}
					&\varphi(N) = \left[\varphi(p_1^{\alpha_{1}})\right]\cdots \left[\varphi(p_s^{\alpha_{s}})\right]=
					\left[p_1^{\alpha_{1}-1}(p_1-1)\right]\cdots \left[p_s^{\alpha_{s}-1}(p_s-1)\right]\\[5pt]
					&\xrightarrow{\text{Closed form}} \prod_{\footnotesize\begin{array}{c}p_i|N \\ p_i \text{ prime}\end{array}}
					{p_i^{\alpha_{i}-1}(p_i-1)} =  \prod_{\footnotesize\begin{array}{c}p_i|N \\ p_i \text{ prime}\end{array}}
					{p_i^{\alpha_{i}}\left(1 - \frac{1}{p_i}\right)}
				\end{align*}
				By multiplying $p_i^{\alpha_i}$ we get the formula as desired.
				\begin{align*}
					\varphi(N) = N \prod_{\footnotesize\begin{array}{c}p_i|N \\ p_i \text{ prime}\end{array}}
					{\left(1-\frac{1}{p_i}\right)}
				\end{align*}
		\end{enumerate}

	\item{\textcolor{for}{\textbf{Problem 5}}}
		\begin{enumerate}[label=(\alph*)]
		\item
			By definition of $\varphi$ we know that $\varphi(N)$ is the number of number less than $N$ and relatively prime to $N$.	
			Assume that order of element $a$ is $d$ by Lagrange's Theorem this $d$ should divivde $\varphi(N)$ so 
			$d \times k = \varphi(N)$, Hence $a^{\varphi(N)} \stackrel{N}{\equiv} a^{d. k} \stackrel{N}{\equiv}1$
		\item
			\begin{align*}
				30 &= 7(4) + 2 \Rightarrow && 2=30 + 7(-4) \\ 
				7 &= 2(3) + 1 \Rightarrow && 1=7  + 2(-3)
			\end{align*}
			\begin{align*}
				1 &= 7 + \left[ 30 + 7(-4) \right] (-3) = 30(-3) + 7(13)
			\end{align*}
			Hence inverse of $7$ modulo $30$ is $13$, i.e $7 \times 13\stackrel{30}{\equiv}1$\\
			\begin{align*}
				&m^7 \stackrel{31}{\equiv} x \Rightarrow \left(m^{7}\right)^{e} \stackrel{31}{\equiv} x^{e} \Rightarrow
				7 \times e \stackrel{\varphi(31)}{\equiv}1 \Rightarrow  7 \times e \stackrel{30}{\equiv}1 \Rightarrow e = 13\\[5pt]
				&\hspace{2.5cm} \left(m^7\right)^{13} \stackrel{31}{\equiv}m \stackrel{31}{\equiv}x^{13}
			\end{align*}
		\end{enumerate}
		\hrule

	\item{\textcolor{for}{\textbf{Problem 6}}}
		\begin{enumerate}[label=(\alph*)]
			\item Assume that $g\in G$ with $\phi(g)=h\in H$ then by definition of identity element we know that $g = 1_{G} *_{G} g$
				Hence $\phi(g) = \phi(1_{G} *_{G} g) = \phi(1_{G}) *_{H} h = h$ the last equation establish that 
				$\phi(1_{G}) = 1_{H}$ $\blacksquare$
			\item 
				\begin{align*}
					1_G = g *_{{}_{G}} g^{-1}\ \xrightarrow{\phi} \phi(1_{{}_{G}}) = 1_{{}_{H}} = \phi(g) *_{{}_{H}} \phi(g^{-1})
					\Rightarrow \phi(g^{-1}) = \left.\phi(g)\right.^{-1}\
				\end{align*}
			\item Obviously $\phi(G)\subseteq H$ only thing we need to show is that $\phi(G)$ is group under the operation of group $H$.
				Just for easier writing assume that operator for $G$ and $H$ are $*$, $\#$, respectively
				\begin{enumerate}[label=(\textbf{\Roman*})]
					\item{\textbf{Closed under Operator}} \\
						Assume for the sake of contradiction that this is not the case, i.e. 
						$h_1, h_2\in \phi(G)$and $g_1, g_2 \in G$ are related elements, but $h_1 \# h_2 \not\in \phi(G)$:
						Because $G$ is group $g_1*g_2\in G$ and Hence $\phi(g_1 * g_2) = h_1 \# h_2$ should be in
						$\phi(G)$ by definition of $\phi$ which contradicts.
					\item {\textbf{Associativity :}} \\
						Is true in $H$ so it is true in $\phi(G)\subseteq H$.
					\item {\textbf{Identity Element :}} By part (a)
					\item {\textbf{Inverse Element :}} By part (b) 
				\end{enumerate}
			\item
				\begin{enumerate}[label=(\textbf{\Roman*})]
					\item{\textbf{Closed under Operator :}} \\
						Assume $a,b \in \ker(\phi)$ then $\phi(a*b)= \phi(a)\ \#\ \phi(b) = 1_{{}_{H}}\ \#\ 1_{{}_{H}} = 1_{{}_{H}}$.
						Thus $a*b\in \ker(\phi)$.
					\item {\textbf{Associativity :}} Quite Obvious.
					\item {\textbf{Inverse Element :}} \\
						If $a\in \ker(\phi)$ then it's inverse is $\phi(a^{-1}) = \left.\phi(a)\right.^{-1}$
						$= \left.1_{{}_{H}}\right.^{-1} = 1_{{}_{H}}$ which means $a^{-1} \in \ker(\phi)$
					\item {\textbf{Identity Element :}} By definition $1_{{}_{G}} \in \ker(\phi)$ 
				\end{enumerate}
				\textbf{Prove that $\ker(\phi)$ is normal subgroup of $G$}: Assume $a\in\ker(\phi)$ and $g\in G$
				\begin{align*}
					\phi(g * a * g^{-1}) = \phi(g)\ \#\ \phi(a)\ \#\ \left.\phi(g)\right.^{-1} = \phi(g)\ \#\ 1_{{}_{H}}\ \#\ \left.\phi(g)\right.^{-1} = 1_{{}_{H}}
				\end{align*}
				Which means $g*a*g^{-1} \in \ker{\phi}$ for every choice of $g\in G$ which by definition means that $\ker(\phi) \triangleleft G$
		\end{enumerate}
	\item{\textcolor{for}{\textbf{Problem 7}}}
	\item{\textcolor{for}{\textbf{Problem 8}}}
\end{itemize}
\end{document}
